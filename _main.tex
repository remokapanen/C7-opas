% Options for packages loaded elsewhere
\PassOptionsToPackage{unicode}{hyperref}
\PassOptionsToPackage{hyphens}{url}
\documentclass[
]{book}
\usepackage{xcolor}
\usepackage{amsmath,amssymb}
\setcounter{secnumdepth}{5}
\usepackage{iftex}
\ifPDFTeX
  \usepackage[T1]{fontenc}
  \usepackage[utf8]{inputenc}
  \usepackage{textcomp} % provide euro and other symbols
\else % if luatex or xetex
  \usepackage{unicode-math} % this also loads fontspec
  \defaultfontfeatures{Scale=MatchLowercase}
  \defaultfontfeatures[\rmfamily]{Ligatures=TeX,Scale=1}
\fi
\usepackage{lmodern}
\ifPDFTeX\else
  % xetex/luatex font selection
\fi
% Use upquote if available, for straight quotes in verbatim environments
\IfFileExists{upquote.sty}{\usepackage{upquote}}{}
\IfFileExists{microtype.sty}{% use microtype if available
  \usepackage[]{microtype}
  \UseMicrotypeSet[protrusion]{basicmath} % disable protrusion for tt fonts
}{}
\makeatletter
\@ifundefined{KOMAClassName}{% if non-KOMA class
  \IfFileExists{parskip.sty}{%
    \usepackage{parskip}
  }{% else
    \setlength{\parindent}{0pt}
    \setlength{\parskip}{6pt plus 2pt minus 1pt}}
}{% if KOMA class
  \KOMAoptions{parskip=half}}
\makeatother
\usepackage{longtable,booktabs,array}
\usepackage{calc} % for calculating minipage widths
% Correct order of tables after \paragraph or \subparagraph
\usepackage{etoolbox}
\makeatletter
\patchcmd\longtable{\par}{\if@noskipsec\mbox{}\fi\par}{}{}
\makeatother
% Allow footnotes in longtable head/foot
\IfFileExists{footnotehyper.sty}{\usepackage{footnotehyper}}{\usepackage{footnote}}
\makesavenoteenv{longtable}
\usepackage{graphicx}
\makeatletter
\newsavebox\pandoc@box
\newcommand*\pandocbounded[1]{% scales image to fit in text height/width
  \sbox\pandoc@box{#1}%
  \Gscale@div\@tempa{\textheight}{\dimexpr\ht\pandoc@box+\dp\pandoc@box\relax}%
  \Gscale@div\@tempb{\linewidth}{\wd\pandoc@box}%
  \ifdim\@tempb\p@<\@tempa\p@\let\@tempa\@tempb\fi% select the smaller of both
  \ifdim\@tempa\p@<\p@\scalebox{\@tempa}{\usebox\pandoc@box}%
  \else\usebox{\pandoc@box}%
  \fi%
}
% Set default figure placement to htbp
\def\fps@figure{htbp}
\makeatother
\setlength{\emergencystretch}{3em} % prevent overfull lines
\providecommand{\tightlist}{%
  \setlength{\itemsep}{0pt}\setlength{\parskip}{0pt}}
\usepackage[]{natbib}
\bibliographystyle{plainnat}
\usepackage{booktabs}
\usepackage{bookmark}
\IfFileExists{xurl.sty}{\usepackage{xurl}}{} % add URL line breaks if available
\urlstyle{same}
\hypersetup{
  pdftitle={C7-opas},
  pdfauthor={Remo Kapanen},
  hidelinks,
  pdfcreator={LaTeX via pandoc}}

\title{C7-opas}
\author{Remo Kapanen}
\date{2025-08-18}

\begin{document}
\maketitle

{
\setcounter{tocdepth}{1}
\tableofcontents
}
\chapter{Oppaasta}\label{oppaasta}

Opas ei ole valmis, sillä lukukausi on tietysti vielä käynnissä. Oppaan pääasiallinen tarkoitus on toimia paikkana, jonne linkkaan lähteet, joista kannattaa opetella asiat ennen Rankin kortteja tai viimeistään silloin kun et ymmärrä jotain korttia.

\chapter{Kirurgia}\label{kirurgia}

\section{Plastiikka- ja yleiskirurgia}\label{plastiikka--ja-yleiskirurgia}

\subsection{01: Suturaatio}\label{suturaatio}

Ei kortteja vielä

\subsection{02: Patit \& iholuomet + Melanooma \& muut ihosyövät}\label{Ihosyovat}

Suosittelen käymään luentodiat \textbf{Patit \& iholuomet} sekä \textbf{Melanooma \& muut ihosyövät} samassa yhteydessä, sillä ne käsittelevät samoja asioita. Kortit on myös tehty sillä logiikalla, että molemmat materiaalit on opiskeltu ennen kortteja.

\begin{itemize}
\item
  Suosittelen myös lukemaan näitä:

  \begin{itemize}
  \tightlist
  \item
    \href{https://www.terveysportti.fi/apps/dtk/ltk/article/duo18406/}{Ihokasvaimen stanssibiopsia vai poisto kokonaan?}
  \item
    \href{https://www.terveysportti.fi/apps/dtk/ltk/article/ykt01386/}{Melanooman Terveysportti-artikkeli}
  \item
    \href{https://www.oppiportti.fi/oppikirjat/pat00688}{Melanooma (Patologia, Duodecim}
  \item
    \href{https://www.terveysportti.fi/apps/dtk/ltk/article/duo11556/}{Basalioomat, okasolusyöpä ja sen esiasteet, miten hoidan?}
  \item
    \href{https://www.terveysportti.fi/apps/dtk/ltk/article/hsu00009}{Kansallinen ei-melanoottisten ihosyöpien hoito-ohjeistus}
  \item
    \href{https://www.oppiportti.fi/oppikirjat/pat00763}{Pehmytkudoskasvaimet}
  \end{itemize}
\end{itemize}

\section{Gastrokirurgia}\label{gastrokirurgia}

\section{Propedeuttinen kirurgia}\label{propedeuttinen-kirurgia}

\section{Thorax- ja verisuonikirurgia}\label{thorax--ja-verisuonikirurgia}

\section{Urologia}\label{urologia}

\chapter{Ortopedia, Traumatologia ja Käsikirurgia}\label{OrtoTraumaKasi}

\section{01: INFO + Luunmurtuman paraneminen ja komplikaatiot}\label{info-luunmurtuman-paraneminen-ja-komplikaatiot}

Diat ja Traumatologia-oppikirjasta kappaleet 21.1-21.8.

\begin{itemize}
\item
  Lisäksi kannattaa katsella näitä:

  \begin{itemize}
  \tightlist
  \item
    \href{https://open.oregonstate.education/aandp/chapter/6-3-bone-structure/}{Luun rakenne-artikkeli}
  \item
    \href{https://say.fi/files/lehtovakkalakaakinen_aitiopaine.pdf}{Aitiopaineoireyhtymä-artikkeli}
  \item
    \href{https://youtu.be/ZD_5Why69IM?si=IJFrbGAw79Cy18Fe}{Murtuman paraneminen-video}
  \item
    \href{https://www.oppiportti.fi/oppikirjat/kia00380?q=avomurtuma}{Avomurtuma-artikkeli}
  \end{itemize}
\end{itemize}

\section{02: Nivelten tutkiminen}\label{nivelten-tutkiminen}

\section{03: Kipsaus}\label{kipsaus}

Diat ja Traumatologia-oppikirjasta kappale 21.10

\begin{itemize}
\tightlist
\item
  Monia korttien kuvia kerätty \href{https://www.oppiportti.fi/oppikirjat/kps00001}{Kipsihoidon perusteet-oppikirjasta (Duodecim)}
\end{itemize}

Diojen lopussa olevat erilliset murtumat ja niiden hoito käsitellään paremmin muissa diapaketeissa, joten kortit sitten niiden yhteydessä.

\chapter{Anestesiologia ja tehohoito}\label{anestesiologia-ja-tehohoito}

\section{Sisäänpääsytentti}\label{Sisaanpaasytentti}

\href{images/Sisäänpääsytentti_ANE.pdf}{Hyshys}

\section{Akuuttihoidon ABCDE}\label{akuuttihoidon-abcde}

Diat sinänsä riittävät, mutta suosittelen katsomaan SketchyIM:n Shock-videot läpi.

\chapter{Radiologia}\label{radiologia}

\chapter{Fysiatria}\label{fysiatria}

\chapter{Palaute}\label{palaute}

2025 on ensimmäinen vuosi, jolloin Ranki ja tämä opas on käytettävissä C7-kurssilla. Olisin hyvin kiitollinen, jos antaisitte vinkkejä sen suhteen, miten voin parantaa projektia. Vinkit ovat erityisen arvokkaita, koska teen kurssin myötä kortteja, joten palautteen myötä voi nopeasti ohjata projektia parempaan suuntaan.

\begin{itemize}
\item
  Neuvoja voi laittaa sähköpostiini (\href{mailto:reakap@utu.fi}{\nolinkurl{reakap@utu.fi}}) tai jos haluaa pysyä anonyyminä, niin voi kirjoittaa \href{https://docs.google.com/forms/d/e/1FAIpQLSericnXGU2U_h7stCFVZ5X0-6Q9BLGdiDugun_Mex3kf_bTpg/viewform?usp=sharing&ouid=112689903880978617225}{Formsin} kautta.
\item
  Samoja reittejä hyödyntäen voi aina myös esittää kysymyksiä, jos sellaisia on.
\end{itemize}

\bibliography{book.bib,packages.bib}

\end{document}
